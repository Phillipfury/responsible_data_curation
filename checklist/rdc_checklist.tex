\section{Responsible Data Curation Checklist for Fairness and Robustness Evaluations}
This checklist translates our HCCV data curation considerations and recommendations into
action items for researchers and practitioners. Presented as a series of questions, these
items are designed to stimulate discussions among data collection teams. The questions
are purposefully worded to avoid binary responses, encouraging open-ended dialogues. The
primary focus of the checklist is to underscore the ethical dimensions and ensure that
teams address concerns encompassing purpose, consent and privacy, as well as diversity.

It is important to engage with the checklist as a preliminary exercise before beginning
data collection. This approach promotes informed decision-making and minimizes risks,
leading to more responsible and reliable outcomes.

Contextual diversity is acknowledged to avoid a one-size-fits-all approach. Moreover,
customization is encouraged, as not all items apply universally; teams should modify or
expand the checklist to align with their context and use case. As with existing AI ethics
checklists, it is important to recognize that the checklist is not a guarantee for
ethical compliance; rather, it functions as a catalyst for discussion and reflection.

We understand that answering these questions is time-consuming, increasing the burden
on data collection teams whose work is already undervalued. Therefore, when navigating
through these lists, priority should be put on items related to the specific domain and
task of interest. The level of engagement needed for each question will invariably
differ. Keep in mind that the questions aim to spur ethical thinking during dataset
development: ``Ethics is often about finding a good or better, but not perfect,
answer''.

%%%%%%%%%%%%%%%%%%%%%%%%%%%%%%%%%%%%%%%%%%%%%%%%%%%%%
%%%%%%%%%%%%%%%%%%%%%%%%%%%%%%%%%%%%%%%%%%%%%%%%%%%%%

%%%%%%%%%%%%%%%%%%%%%%%%%%%%%%%%%%%%%%%%%%%%%%%%%%%%%
% Checklist: Purpose
%%%%%%%%%%%%%%%%%%%%%%%%%%%%%%%%%%%%%%%%%%%%%%%%%%%%%
\subsection{Purpose}
The questions in this section focus on eliciting strategies for curating HCCV
evaluation datasets specifically for fairness and robustness assessments. They seek
alignment with objectives and inquire about factors known to influence these assessments
to ensure comprehensive evaluations. Moreover, the questions aim to assist in formulating
clear dataset purpose statements, preventing ambiguity and misuse of data, as well as
exploring external validation to enhance transparency and accountability.

\paragraph{Dataset Development Strategy}
\begin{itemize}
    \item Can you provide details about your strategy for developing a new dataset tailored
    specifically for conducting fairness and robustness assessments in the context of HCCV?
    How do you plan to ensure that this dataset is aligned with the objectives of evaluating
    fairness and robustness?
    \item Can you elaborate on the factors your dataset will encompass to comprehensively
    enable fairness and robustness evaluations for HCCV models? How do you intend to capture
    the primary factors, including data subjects, instruments, and environments, that
    influence these evaluations?
\end{itemize}

\paragraph{Dataset Purpose Statement} 
\begin{itemize}
    \item Can you provide details about your plan to formulate a comprehensive dataset
    purpose statement? How will this statement effectively communicate the core
    motivations driving, e.g., data collection, outline the intended dataset
    composition, specify permissible uses of the data, and identify the specific
    audience you aim to serve with the dataset?
    \item Can you elaborate on your strategy for ensuring the accuracy and ethical
    alignment of your dataset's purpose statement? How do you plan to externally
    validate the content and ethical considerations of the statement?
    \item Can you provide insights into the benefits and implications of submitting
    your dataset's purpose statement as part of a research study proposal in the format
    of a registered report for your project?
\end{itemize}
%%%%%%%%%%%%%%%%%%%%%%%%%%%%%%%%%%%%%%%%%%%%%%%%%%%%%
%%%%%%%%%%%%%%%%%%%%%%%%%%%%%%%%%%%%%%%%%%%%%%%%%%%%%

%%%%%%%%%%%%%%%%%%%%%%%%%%%%%%%%%%%%%%%%%%%%%%%%%%%%%
% Checklist: Consent and Privacy
%%%%%%%%%%%%%%%%%%%%%%%%%%%%%%%%%%%%%%%%%%%%%%%%%%%%%
\subsection{Consent and Privacy} 
The questions in this section explore informed consent, legal compliance, and privacy
protection measures within anonymization strategies. The questions emphasize clarity
and voluntariness in consent processes to prevent coercion or misuse of data. Moreover,
they attempt to elicit strategies for explaining data collection purposes, consent
revocation, and accommodating diverse participation circumstances. Furthermore, the
questions seek insights into addressing anonymization challenges, aiming to prevent
re-identification risks, unauthorized exposure, and legal noncompliance, while
preserving data utility and protecting data subjects' rights.

\paragraph{Informed and Voluntary Consent} 
\begin{itemize}
    \item Can you elaborate on your approach to ensuring that you secure explicit,
    voluntary, and informed consent from all individuals who either appear in the
    dataset or can be discerned from it? How do you plan to handle consent for data
    annotators who may have disclosed personal information for the purposes of
    quantifying and addressing annotator perspectives and bias?
    \item Can you provide a comprehensive explanation of your strategy for conveying
    the purpose of data collection to the subjects? How do you intend to emphasize the
    utilization of their data, which includes various types of information such as
    facial, body, biometric images, as well as information about themselves and their
    environment, all in the context of assessing the fairness and robustness of HCCV
    systems?
    \item In what ways will you incorporate consent forms that are composed in plain
    language to enhance the understanding of AI technologies? How do you plan to make
    sure these forms effectively convey the intricacies of data usage?
    \item How do you plan to inform data subjects about their ability to withdraw
    consent at any given point during, or after, the data collection process? Can you
    provide details about the mechanisms you will have in place for facilitating this?
    \item Please provide insight into your strategy for collecting data from
    individuals below the age of majority or vulnerable individuals. How will you seek
    both guardian consent and voluntary informed assent in such cases?
    \item How do you plan to evaluate vulnerability along a continuous spectrum,
    taking into account contextual factors and recognizing that vulnerability is not
    solely binary or based solely on group affiliations, but can also be influenced
    by specific situations or circumstances?
    \item Can you also provide details about how you will consider the circumstances
    of participation, which might include the potential need for participatory design,
    assurances of compensation, provision of educational materials, and safeguards
    against authoritative structures? How will you address these various aspects in
    your approach?
    \item How do you intend to ensure that vulnerable individuals have a comprehensive
    understanding of the data usage and willingly provide informed assent? Can you
    outline the specific measures you intend to implement for this purpose?
    \item Can you elaborate on how you will respect the decision of a vulnerable
    individual who expresses dissent, regardless of the preferences of their guardian?
\end{itemize}

\paragraph{Consent Revocation Mechanisms} 
\begin{itemize}
    \item How do you plan to integrate mechanisms that allow data subjects to easily
    withdraw their consent? Can you provide specifics on how this process will be
    designed and executed?
    \item Can you provide insights into the benefits and implications of implementing
    dynamic consent mechanisms that utilize personalized communication interfaces?
    How do you intend to ensure that these mechanisms adapt to the preferences and
    needs of individual data subjects?
    \item How do you intend to enable data subjects to actively participate in research
    activities and manage their consent preferences? Can you provide more details about
    the tools or processes you plan to put in place to achieve this?
    \item In what ways will you explore the feasibility of online platforms for consent
    management that are user-friendly and minimize complexity for data subjects? What
    steps will you take to ensure easy accessibility?
    \item Can you provide insights into the options you will provide to data subjects
    for granting consent? How will you offer choices between blanket consent,
    case-by-case selection, or opt-in based on specific data usage?
    \item Can you elaborate on your considerations regarding the formation of a
    steering board or charitable trust composed of representative subjects from the
    dataset? How do you envision this entity contributing to decision-making processes?
    \item How do you plan to empower data subjects to actively participate in decisions
    concerning the usage of their data? What mechanisms or channels will you establish
    to facilitate this involvement?
    \item Can you provide information about the method you will offer data subjects to
    easily and promptly revoke their consent? How will you ensure that this process is
    straightforward and accessible?
    \item How do you intend to address varying levels of technological know-how and
    internet access among data subjects? Can you detail the measures you will take to
    accommodate these variations?
    \item What alternatives do you plan to offer for revoking consent that do not rely
    solely on online-based processes? How will you ensure that individuals with
    different needs and preferences can effectively revoke their consent?
    \item How do you plan to assess the practicality and suitability of the chosen
    mechanisms for consent revocation, taking into account the expected dataset size
    and the resources available to you? What criteria will you use to evaluate their
    effectiveness?
\end{itemize}

\paragraph{Country of Residence Information} 
\begin{itemize}
    \item How do you plan to address the fact that anonymization measures might not
    universally meet legal requirements in specific regions, necessitating additional
    considerations? Can you provide insights into your strategy for ensuring legal
    compliance while implementing anonymization?
    \item Can you elaborate on your approach to collecting information about the
    country of residence for each individual in your dataset? How do you intend to use
    this information to ensure legal compliance and address potential privacy concerns?
    \item How do you plan to familiarize yourself with the data protection laws that
    are applicable in the countries of residence of your data subjects? Can you provide
    details about your process for gaining this understanding and how you will apply
    it to your data curation project?
    \item How do you intend to prioritize safeguarding data subjects' rights as
    stipulated by the data protection laws in their respective countries? What steps
    will you take to ensure that the creation and utilization of the dataset strictly
    adhere to the relevant data protection regulations? Can you provide specifics
    about the measures you will put in place to achieve this?
    \item What mechanisms do you intend to implement to ensure the adaptability of
    your dataset management strategy to changing legislative requirements? Can you
    provide details about how you will monitor and accommodate legislative changes
    in your dataset management approach? Can you provide insights into how you will
    strike a balance between maintaining compliance and effective dataset management
    in dynamic legal environments?
\end{itemize}

\paragraph{Privacy-Sensitive Image Regions and Metadata} 
\begin{itemize}
    \item How do you plan to implement measures that effectively safeguard against
    re-identification risks, encompassing singling out, linkability, and inference,
    within your anonymization approach?
    \item Can you elaborate on your strategy for redacting all image regions that could
    inadvertently disclose privacy-related information? How do you intend to
    comprehensively identify and address these regions?
    \item Can you elaborate on your strategy for the removal of elements such as body
    parts, clothing, and accessories for nonconsenting subjects to enhance privacy
    protection? Can you provide more details about the considerations and methods
    involved in this process?
    \item Can you elaborate on your strategy for the removal of text (possibly
    excluding copyright owner information) from the dataset's images to enhance
    privacy protection? Can you provide more details about the considerations and
    methods involved in this process?
    \item Can you explain your plan for empirically validating the chosen anonymization
    methods? How will you assess the methods' effectiveness in mitigating
    re-identification risks while preserving the utility of the data?
    \item Can you provide details about how human annotators will be engaged in the
    creation and verification of privacy leaking image region proposals for
    anonymization purposes? How will you ensure accuracy and consistency in this process?
    \item Can you provide details about how you intend to align region proposals
    predicted by algorithms with human judgment, addressing any potential failures or
    biases? Can you describe your strategy for maintaining a sensitive approach to
    these factors?
    \item What steps will you take to address jurisdiction-specific requirements that
    might necessitate human-generated proposals for biometric identifiers in order to
    comply with legal and regulatory standards?
    \item Can you elaborate on the measures you will take to prevent image metadata
    from inadvertently revealing unauthorized identifying information? How will you
    ensure that metadata remains privacy-conscious?
    \item How will you identify specific metadata elements that you intend to retain
    to ensure a comprehensive understanding during the evaluation process? Can you
    provide examples of the types of metadata you plan to retain for this purpose?
    \item How do you plan to replace or remove sensitive information within metadata
    while retaining its usefulness for fairness and robustness analyses? Can you
    provide insights into your approach for striking a balance in this regard?
\end{itemize}
%%%%%%%%%%%%%%%%%%%%%%%%%%%%%%%%%%%%%%%%%%%%%%%%%%%%%
%%%%%%%%%%%%%%%%%%%%%%%%%%%%%%%%%%%%%%%%%%%%%%%%%%%%%

%%%%%%%%%%%%%%%%%%%%%%%%%%%%%%%%%%%%%%%%%%%%%%%%%%%%%
% Checklist: Diversity
%%%%%%%%%%%%%%%%%%%%%%%%%%%%%%%%%%%%%%%%%%%%%%%%%%%%%
\subsection{Diversity}
The questions in this section revolve around obtaining accurate image annotations
related to identity, phenotype, environmental factors, and instruments, while upholding
inclusivity, sensitivity, and privacy. Additionally, the questions attempt to elicit
strategies for documenting identity, ensuring fair compensation, and effective
(anonymous) communication.

\paragraph{Self-Reported Annotations}
\begin{itemize}
    \item How do you plan to acquire annotations for images directly from the data
    subjects, leveraging their self-awareness and contextual knowledge to enhance the
    accuracy and quality of annotations? Can you elaborate on the methods and
    strategies you intend to use for this purpose?
    \item Can you elaborate on your strategy for addressing biases and ensuring careful
    handling when inferring labels about individuals? Can you provide reasoning as to
    why labels about individuals will be inferred as opposed to being self-identified?
    How will you actively mitigate potential biases that may arise during the labeling
    process?
    \item How do you intend to consider the implications of inferred labels, for
    example, in relation to data access request rights?
\end{itemize}

\paragraph{Versatile and Inclusive Response Options}
\begin{itemize}
    \item How do you plan to enhance the accuracy and nuance of identity information
    collection by providing respondents with both closed-ended and open-ended response
    choices? Can you elaborate on your strategy for using open-ended responses to
    gather more detailed and comprehensive data?
    \item How do you intend to ensure inclusivity and prevent any potential
    implications of exclusion in the response choices you offer?
    \item Can you elaborate on your preparedness to manage the coding and analysis
    effort required for processing open-ended responses? What effective strategies
    do you plan to implement for managing and analyzing the data collected from
    open-ended questions? How will you handle the potential complexities and variations
    that can arise from these responses, ensuring that the insights and information
    derived can be accurately captured and utilized?
\end{itemize}

\paragraph{Dynamic Nature of Identity}
\begin{itemize}
    \item How do you plan to collect self-identified information on a per-image basis,
    accounting for the fact that identity is intrinsically contextual and temporal? Can
    you elaborate on your strategy for capturing nonstatic aspects of identity?
    \item Can you elaborate on your strategy for enabling data subjects to freely
    choose multiple identity categories without imposing any limitations? How will you
    ensure that subjects have the flexibility to express their identity in a
    comprehensive and unrestrictive manner?
    \item How do you intend to address potential requests for per-image updates to
    self-identified information provided by subjects over time, respecting their
    autonomy? What factors have you considered in relation to the potential effects of
    permitting updates?
\end{itemize}

\paragraph{Demographic Information}
\begin{itemize}
  \item How do you plan to collect precise biological age in years from data subjects
  to ensure an accurate representation of their age?
  \item Can you elaborate on your approach to gathering pronoun information from data
  subjects to enhance gender inclusivity and mitigate the risk of misgendering? How
  will you ensure that respondents feel comfortable providing this information?
  \item Can you explain your strategy for gathering consistent ancestry information
  from data subjects? How will you approach the collection of this information in a
  sensitive and inclusive manner?
  \item How do you intend to offer the option for data subjects not to disclose their
  sensitive attributes if they choose not to? Can you provide more details about how
  you will handle the sensitivity and privacy of these attributes?
\end{itemize}

\paragraph{Sensitive Attributes in Aggregate}
\begin{itemize}
  \item How do you plan to collect voluntarily disclosed sensitive attributes such as
  disability and pregnancy status? Can you elaborate on your approach to respecting the
  willingness of data subjects to provide these details?
  \item Can you provide insight into your strategy for reporting sensitive attributes,
  such as disability and pregnancy status, in aggregate data while safeguarding
  subjects' safety and privacy? How do you intend to ensure that individual identities
  are protected?
  \item Can you elaborate on your approach to relying on credible and appropriate
  sources for the categorization and definitions of sensitive attributes like
  disability or difficulty? How will you account for the potential variations in
  these definitions based on cultural, identity, and historical contexts?
\end{itemize}

\paragraph{Phenotypic and Neutral Performative Features}
\begin{itemize}
  \item How do you plan to collect phenotypic attributes, encompassing characteristics
  such as skin color, eye color, hair type, hair color, height, and weight? Can you
  provide insights into your strategy for obtaining these attributes in a sensitive
  and comprehensive manner?
  \item Can you elaborate on your approach to collecting a diverse range of neutral
  performative features, including aspects such as facial hair, hairstyle, cosmetics,
  clothing, and accessories? How do you intend to ensure inclusivity and accuracy in
  capturing these features?
\end{itemize}

\paragraph{Environment and Instrument Details}
\begin{itemize}
  \item How do you plan to gather data on environment-related factors, which encompass
  details such as image capture time, season, weather, ambient lighting, scene,
  geography, camera position, and camera distance? Can you provide insights into your
  strategy for capturing these factors accurately and comprehensively?
  \item Can you elaborate on your approach to collecting instrument-related factors
  concerning the imaging devices used, including aspects such as lens, sensor,
  stabilization, flash usage, and post-processing software? How do you intend to
  ensure accuracy in capturing these details?
  \item How do you plan to obtain environment- and instrument-related information?
  Can you provide more details about the methods you will use, such as self-reporting,
  annotator estimation, and sourcing information from Exif metadata? How will you
  leverage contextual knowledge from image subjects to enhance data quality?
  \item Can you explain your approach to handling information such as precise
  geolocation and user-added details in Exif metadata that might contain personally
  identifying information? How will you ensure compliance with copyright regulations
  (if applicable) while maintaining privacy and adhering to ethical considerations?
\end{itemize}

\paragraph{Annotators as Contributors}
\begin{itemize}
  \item How do you plan to document the identities of data annotators, including
  capturing demographic details such as pronouns, age, and ancestry? Can you provide
  insights into your strategy for gathering and preserving this information while
  respecting privacy and ensuring transparency?
  \item Can you elaborate on your approach to highlighting the contributions of
  annotators beyond data labeling in the dataset documentation after the curation
  process? How do you intend to accurately represent the multifaceted roles and
  contributions of annotators?
  \item How do you plan to report the demographic information of annotators to
  analyze potential sources of bias in dataset annotations? Can you provide more
  details about your proposed approach for conducting this analysis while ensuring
  privacy and ethical considerations?
\end{itemize}

\paragraph{Fair Treatment and Compensation}
\begin{itemize}
  \item How do you plan to ensure that all contributors receive compensation that
  exceeds the minimum hourly wage of their respective country or jurisdiction of
  residence? Can you provide insights into your compensation strategy to ensure fair
  and ethical remuneration?
  \item Can you elaborate on your approach to exploring alternative payment models,
  such as compensation based on the average hourly wage? How do you intend to determine
  a compensation structure that is both fair and reflective of contributors' efforts?
  \item How will you establish direct communication channels between dataset creators
  and contributors? Can you provide more details about the methods you intend to
  implement for effective and transparent communication?
  \item What communication methods do you plan to explore that maintain the anonymity
  of contributors? Can you provide insights into your approach to balancing communication
  and privacy needs, such as using anonymous feedback forms?
  \item Can you provide information about your strategy for developing clear and
  accessible plain language guides to facilitate various tasks, such as image submission
  and data annotation? How do you plan to ensure that these guides effectively assist
  contributors?
  \item How do you intend to ensure that contributors from diverse backgrounds can easily
  understand and follow any instructions provided? Can you elaborate on your approach to
  promoting inclusivity and accessibility in your communication and guidelines?
  \item Can you provide details about how you plan to subject your recruitment and
  compensation procedures to ethics review? What steps will you take to ensure that
  your procedures align with ethical considerations and best practices?
\end{itemize}
%%%%%%%%%%%%%%%%%%%%%%%%%%%%%%%%%%%%%%%%%%%%%%%%%%%%%
%%%%%%%%%%%%%%%%%%%%%%%%%%%%%%%%%%%%%%%%%%%%%%%%%%%%%